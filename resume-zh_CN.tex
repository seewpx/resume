% !TEX TS-program = xelatex
% !TEX encoding = UTF-8 Unicode
% !Mode:: "TeX:UTF-8"

\documentclass{resume}
\usepackage{zh_CN-Adobefonts_external} % Simplified Chinese Support using external fonts (./fonts/zh_CN-Adobe/)
% \usepackage{NotoSansSC_external}
% \usepackage{NotoSerifCJKsc_external}
% \usepackage{zh_CN-Adobefonts_internal} % Simplified Chinese Support using system fonts
\usepackage{linespacing_fix} % disable extra space before next section
\usepackage{cite}

\begin{document}
\pagenumbering{gobble} % suppress displaying page number

\name{许伟平}

\basicInfo{
  \email{weiping\_xu@aliyun.com} \textperiodcentered\ 
  \phone{(+86) 130-182-50670} %\textperiodcentered\ 
  %\linkedin[billryan8]{https://www.linkedin.com/in/billryan8}
  }
 
\section{\faGraduationCap\  教育背景}
\datedsubsection{\textbf{电子科技大学}, 成都}{2018 -- 至今}
\textit{在读硕士研究生}\ 仪器仪表工程, 预计 2021 年 6 月毕业
\datedsubsection{\textbf{电子科技大学}, 成都, 四川}{2014 -- 2018}
\textit{学士}\ 自动化(大类),测控技术与仪器

\section{\faCogs\ 技术能力}
% increase linespacing [parsep=0.5ex]
\begin{itemize}[parsep=0.5ex]
  \item 编程语言熟练程度:python > C++ >= bash/shell > matlab >= C\#
  \item 开发平台:Linux 
  \item 深度学习库熟悉程度:pytorch > keras >= tensorflow
  \item 熟悉linux系统,熟练使用Gentoo软件自动化编译构建系统
  \item 参与拥抱开源世界,为TeXmacs,Gentoo官方库提交BUG、PR,Gentoo发行版社区wiki的志愿维护者。
  \item linux运维能力,管理教研室Linux服务器,拥有自己的编译服务器, 网站和vps。
  \item 了解嵌入式开发,PCB画板投板。
\end{itemize}


\section{\faUsers\ 项目经历}
\datedsubsection{\textbf{无损检测与结构监测研究中心} 自动化学院}{2018年9月 -- 至今}
\role{国家自然科学基金重大仪器专项}{钢轨接触疲劳及裂纹多物理高速巡检监测技术攻关和仪器研发}
缺陷检测功能集成,探伤设备便携化,软件环境搭建,功能协助开发
\begin{itemize}
  \item 提高检测精度和速度;
  \item 配合研究电磁脉冲激励红外热成像的速度效应;
  \item 读取视频流,检测并标注缺陷位置(画框),实时显示并统计。
\end{itemize}



\section{\faCogs\ 科研能力}
% increase linespacing [parsep=0.5ex]
\datedsubsection{\textbf{一、发表\ Pattern Deep Region Learning for Crack Detection in Thermography Diagnosis System } metals}{2018年6月 -- 2018年9月}
\role{第一作者}{SCI三区}
关于电磁脉冲激励红外热成像静态图像表面缺陷检测,从Faster R-CNN改进,
基于数据特点改进数据增强方法,调整网络结构,加入计算机视觉注意力机制(nolocal),
在ResNet的ShortCut加入约束,在features的channels间加入稀疏优化,有效提高检测精度和召回率。
\datedsubsection{\textbf{二、研究\ 基于背景建模的缺陷检测方法} }{2018年9月 -- 2019年6月}
(研一上课期间) 由于数据样本小,探索从背景出发,利用自编码器(AE,VAE,staged-AE/VAE,etc.)训练缺陷异常检测网络,
构建基于平均值方差值和协方差值多值加权相似最大化损失函数,弥补自编码器输出模糊,像素位置不匹配对应的问题,
对碳纤维,纺织布等结构性数据效果明显,对缺陷信息易被淹没的电磁脉冲红外热成像也有一定效果。
\datedsubsection{\textbf{三、正在编写第二篇论文\ 电磁脉冲激励红外热成像时间上下文视频表面缺陷检测} }{2019年7月 -- 至今}
(最近)电磁脉冲激励红外热成像时间上下文视频表面缺陷检测,采用One-stage,点面检测(二维高斯加权),模糊匹配,低秩矩阵约束,
改进的卷积化LSTM。对应于电磁脉冲激励红外热成像过程中的热扩散现象,表面缺陷裂纹等小而难以辨认的特点,小样本数据的特点。
% Reference Test
%\datedsubsection{\textbf{Paper Title\cite{zaharia2012resilient}}}{May. 2015}
%An xxx optimized for xxx\cite{verma2015large}
%\begin{itemize}
%  \item main contribution
%\end{itemize}
\section{\faHeartO\ 研究生学业课程}
信号处理方法及应用、现代信号处理、计量方法与误差理论、随机过程及应用、矩阵理论、计算机视觉、现代时域测试、现代检测技术

\section{\faHeartO\ 获奖情况}
\datedline{\textit{二等奖学金}}{研究生二年级}
\datedline{二等奖学金、一等奖学金}{本科}
\datedline{微芯杯电子设计竞赛一等奖}{本科}

\section{\faInfo\ 其他}
% increase linespacing [parsep=0.5ex]
\begin{itemize}[parsep=0.5ex]
  \item 个人网站: https://ai.seewpx.me/nextcloud/
  \item GitHub: https://github.com/seewpx
  \item 语言: 英语 - 熟练(雅思 备考中,四六级已通过)
\end{itemize}

%% Reference
%\newpage
%\bibliographystyle{IEEETran}
%\bibliography{mycite}
\end{document}
